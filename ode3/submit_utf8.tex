%%%%%%%%%%%%%%%%%%%%%%%%%%%%%%%%%%%%%%%%%%%%%%%%%%%%%%%%%%%%%%%%%%%%%%
%%                               									%%
%%  		【2014年フィードバック制御 HW用texファイル】					%%
%%          自由に改変してもらって構いません.美しい文章を目指しましょう.    %%                                          
%%     																%%
%%                                                                  %%
%%%%%%%%%%%%%%%%%%%%%%%%%%%%%%%%%%%%%%%%%%%%%%%%%%%%%%%%%%%%%%%%%%%%%%


\documentclass[notitlepage,11pt]{jarticle}

\usepackage{FCHW_utf8}

\begin{document}

%%%%%%%%%%%%%%%%%%%%%%%%%%%%%%%%%%%%%%%%%%%%%%%%%%%%%%
%		 			タイトル部分開始 					 %
%		学籍番号,氏名,提出日を忘れずに記載すること        %
%%%%%%%%%%%%%%%%%%%%%%%%%%%%%%%%%%%%%%%%%%%%%%%%%%%%%%

\begin{center}
{\bf \LARGE フィードバック制御 Homework \#1} \\
\vspace{15pt}
{\bf \Large 13B*****} 				% 学籍番号 
{\bf \Large ****}\\				% 氏名
\vspace{15pt}
{\large 共同制作者:** **} \\	% 共同制作者名.誰かと一緒に行った場合は必ず記載.
\vspace{15pt}
{\large 提出日: 2014年10月15日} 	% 提出日

\end{center}

%%%%%%%%%%%%%%%%%%%%%%%%%%%%%%%%%%%%%%%%%%%%%%%%%%%%%%
%		 		    タイトル部分終了 					 %
%%%%%%%%%%%%%%%%%%%%%%%%%%%%%%%%%%%%%%%%%%%%%%%%%%%%%%

\vspace{2zw}

%%%%%%%%%%%%%%%%%%%%%%%%%%%%%%%%%%%%%%%%%%%%%%%%%%%%%%
%		 		    本文開始		 					 %
%%%%%%%%%%%%%%%%%%%%%%%%%%%%%%%%%%%%%%%%%%%%%%%%%%%%%%

\question{HW1-1}1次系の応答について考察する. 1次系の伝達関数は,時定数$T$とゲイン$K$を用いると,
\begin{eqnarray}
\frac{K}{Ts + 1} \label{eq:1} % \labelを用いることで,文章内でこの式を参照することができる.
\end{eqnarray}
と表すことができる.\req{eq:1}において,$T = 1$, $K = 1$としたときのボード線図を\rfig{fig:1}に示す.

\begin{figure}[!h]
	\centering
	\includegraphics[width=100mm,clip]{bode1st.eps}
	\flushright
	\vspace{-15pt}
	\caption{1次系のボード線図}
	 \label{fig:1}
\end{figure}

\question{HW1-2}2つの図を並べることもできる.

\begin{figure}[htbp]
 \begin{minipage}{0.5\hsize}
  \begin{center}
   \includegraphics[width=70mm]{bode1st.eps}
  \end{center}
  \caption{1つめの図}
  \label{fig:2}
 \end{minipage}
 \begin{minipage}{0.5\hsize}
  \begin{center}
   \includegraphics[width=70mm]{bode1st.eps}
  \end{center}
  \caption{2つめの図}
  \label{fig:3}
 \end{minipage}
\end{figure}

%%%%%%%%%%%%%%%%%%%%%%%%%%%%%%%%%%%%%%%%%%%%%%%%%%%%%%
%		 		    本文終了		 					 %
%%%%%%%%%%%%%%%%%%%%%%%%%%%%%%%%%%%%%%%%%%%%%%%%%%%%%%

\end{document}